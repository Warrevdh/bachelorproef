\chapter{Analyse van resultaten}
\label{ch:analyse-testen}

De analyse van de Proof of Concept (PoC) is een essentieel onderdeel van de bachelorproef. Deze fase biedt waardevolle inzichten in de prestaties, bruikbaarheid en algehele effectiviteit van de geïntegreerde oplossing. Gedurende dit hoofdstuk worden de evaluatiemethoden, de verzamelde data en de interpretatie daarvan uitvoerig besproken.

\section{Evaluatiemethoden}

De PoC is beoordeeld aan de hand van diverse methoden die samen een uitgebreid beeld geven van de functionaliteit en gebruikerservaring:
\begin{itemize}
    \item \textbf{Gebruikerstesten}: Verschillende sessies met eindgebruikers zijn uitgevoerd om de interactie met de geïntegreerde kalenderoplossing te observeren. De testen waren ontworpen om realistische scenario's na te bootsen die gebruikers in hun dagelijkse werk kunnen tegenkomen. Hierbij werden aspecten zoals de intuïtiviteit van de gebruikersinterface, de gemakkelijkheid van het uitvoeren van taken en de nauwkeurigheid van de dataoverdracht tussen verschillende kalendersystemen beoordeeld. Aandachtspunten waren onder meer de snelheid van uitvoering, gebruiksgemak en het registreren van eventuele gebruikersfouten of frustraties.
    \item \textbf{Prestatiemetingen}: Om een objectieve maatstaf van de technische prestaties te verkrijgen, zijn systematische metingen uitgevoerd betreffende de laadtijden, responsiviteit van de interface en algemene stabiliteit van de applicatie. Deze metingen werden verricht met behulp van browsergebaseerde ontwikkelaarstools die gedetailleerde inzichten geven in de verwerkingstijd en het geheugengebruik. Serverlogbestanden waren ook een belangrijke bron van informatie, vooral voor het identificeren van backend problemen zoals vertragingen bij data-aanvragen of fouten bij API-interacties.
\end{itemize}

\section{Data-analyse}

De verzamelde gegevens uit de gebruikerstesten en technische evaluaties zijn grondig geanalyseerd. De focus lag op het identificeren van duidelijke trends en het lokaliseren van probleemgebieden binnen de PoC. Hier zijn enkele van de meest opvallende bevindingen:

\begin{itemize}
    \item \textbf{Gebruiksvriendelijkheid}: Over het algemeen werd de gebruikersinterface goed ontvangen door de testers. Veel gebruikers prezen de naadloze integratie van verschillende kalenderdiensten en vonden de interface intuïtief en gemakkelijk in gebruik. Niettemin waren er enkele opmerkingen over moeilijkheden bij het overschakelen tussen verschillende kalenderweergaven, vooral wanneer gebruikers probeerden complexe planningstaken uit te voeren. Dit suggereert dat er ruimte is voor verbetering in de navigatie en mogelijkheden voor gebruikersaanpassing.
    \item \textbf{Prestatieproblemen}: Hoewel de applicatie over het algemeen stabiel presteerde, waren er incidentele prestatieproblemen, vooral bij het laden van grote hoeveelheden data van de Calendly API. Deze problemen resulteerden in merkbare vertragingen, wat een negatieve invloed had op de gebruikerservaring. Dit benadrukt de noodzaak voor verdere optimalisatie van de datahandling en mogelijk een herziening van de manier waarop data worden opgehaald en gecached.
\end{itemize}

\section{Conclusie}

De uitgevoerde analyses bieden belangrijke inzichten in zowel de sterke als de verbeterpunten van de PoC. Door voort te bouwen op deze bevindingen, kunnen gerichte aanpassingen en optimalisaties worden doorgevoerd die de effectiviteit, efficiëntie en de algehele gebruikerservaring van de geïntegreerde kalenderoplossing zullen verhogen.
