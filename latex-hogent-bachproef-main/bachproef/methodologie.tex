%%=============================================================================
%% Methodologie
%%=============================================================================

\chapter{\IfLanguageName{dutch}{Methodologie}{Methodology}}%
\label{ch:methodologie}


\section{Literatuurstudie}
De literatuurstudie vormde de fundamenten van dit onderzoek. Een zorgvuldige evaluatie van de beschikbare literatuur was essentieel om een grondig begrip van de bestaande technologieën en methoden die relevant zijn voor de integratie van kalender-API's te verkrijgen. Academische artikelen, industriële witboeken, technische documentatie van de API's, en recente case studies werden systematisch verzameld en geanalyseerd. Deze literatuur hielp bij het identificeren van de hiaten in de huidige kennis en technologieën, en stelde de basis van het onderzoek vast waarop verdere experimenten en ontwikkelingen konden worden gebouwd.

\section{Opzetten van de Proof of Concept (PoC)}
De ontwikkeling van de Proof of Concept (PoC) was een cruciale stap in het praktisch testen van de theorieën en inzichten verkregen uit de literatuurstudie. Dit proces begon met het definiëren van de functionele en technische vereisten voor de PoC. Vervolgens werd de architectuur van de applicatie ontworpen, waarbij rekening werd gehouden met de integratie van meerdere kalender-API's, waaronder Google Calendar, Calendly en Microsoft Calendar. De keuze voor specifieke technologieën en frameworks werd geleid door hun vermogen om flexibel te integreren en te schalen. Het opzetten van de PoC omvatte zowel front-end als back-end ontwikkeling, waarbij gebruik gemaakt werd van React.js en Node.js om een responsieve en interactieve applicatie te bouwen.

\section{Analyse van de Resultaten}
Na het implementeren van de PoC werd een gedetailleerde analyse uitgevoerd. De prestaties, functionaliteit, en gebruikersinteracties van de geïntegreerde kalenderoplossing werden geëvalueerd door zowel kwantitatieve als kwalitatieve methoden. Kwantitatieve data zoals laadtijden, responsiviteit en foutenpercentages werden verzameld via geautomatiseerde tests en logboeken. Kwalitatieve feedback werd verkregen via gebruikerstesten en enquêtes. Deze informatie werd gebruikt om de effectiviteit van de technische implementatie te beoordelen en om te begrijpen hoe gebruikers de integratie ervaren.

\section{Conclusie}
De conclusies van deze bachelorproef zijn getrokken op basis van een uitgebreide analyse van de verzamelde gegevens uit de PoC. Deze conclusies bieden inzicht in de toepasbaarheid van de onderzochte technologieën binnen bestaande systemen en hun potentieel om de gebruikerservaring te verbeteren. De resultaten suggereren dat de geïntegreerde kalenderoplossing zowel effectief als efficiënt de gegevensbeheer- en interactietaken van gebruikers kan faciliteren. Aanbevelingen voor verdere verbeteringen en aanpassingen werden ook geformuleerd, gericht op het verhogen van de algehele prestaties en gebruikerstevredenheid.

