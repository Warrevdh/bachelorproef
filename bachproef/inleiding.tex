%%=============================================================================
%% Inleiding
%%=============================================================================

\chapter{\IfLanguageName{dutch}{Inleiding}{Introduction}}%
\label{ch:inleiding}

In een wereld waar afspraken een cruciale rol spelen in zowel het persoonlijke als zakelijke leven, is kalendersoftware onmisbaar geworden. Of het nu gaat om het plannen van vergaderingen, het bijhouden van deadlines of het beheren van persoonlijke verplichtingen, betrouwbare en efficiënte kalendersystemen zijn essentieel. Echter, door de verscheidenheid aan beschikbare kalendersoftware, elk met hun eigen unieke kenmerken en integratiemethoden, staan gebruikers en bedrijven voor de uitdaging om deze systemen op een naadloze manier te combineren en beheren.

In deze bachelorproef wordt de focus gelegd op de integratie van verschillende kalender-API's in softwareapplicaties. Het doel is om te onderzoeken hoe deze integraties de gebruikerservaring beïnvloeden en om innovatieve methoden te ontwikkelen die de efficiëntie en gebruiksvriendelijkheid verbeteren. Door de toenemende behoefte aan gecentraliseerde en efficiënte oplossingen voor kalenderbeheer, vormt dit onderzoek een belangrijke stap richting de optimalisatie van kalendersystemen in een breed scala aan toepassingen.

In samenwerking met TurnUp, een bedrijf dat zich inzet voor het oplossen van het wereldwijde probleem van no-shows en het bevorderen van een cultuur van punctualiteit en betrouwbaarheid, richt dit onderzoek zich op het ontwikkelen van een oplossing die niet alleen technische uitdagingen aanpakt, maar ook bijdraagt aan een verbeterde gebruikservaring.

\section{\IfLanguageName{dutch}{Probleemstelling}{Problem Statement}}%
\label{sec:probleemstelling}
\subsection{TurnUp}
TurnUp heeft als missie om de manier waarop de wereld omgaat met afspraken te revolutioneren door het wereldwijde probleem van no-shows aan te pakken \autocite{TurnUp2024}. Het bedrijf streeft ernaar om de uitdagingen en inefficiënties die gepaard gaan met gemiste afspraken te elimineren, en om zowel individuen als bedrijven te helpen het maximale uit geplande afspraken te halen.

Door middel van innovatieve oplossingen en geavanceerde technologie wil TurnUp een wereldwijde cultuur van betrouwbaarheid, punctualiteit en naadloze afsprakenervaringen creëren. Het bedrijf brengt een diverse gemeenschap van gebruikers, bedrijven en dienstverleners samen om een netwerk te vormen waarin afspraken niet slechts momenten zijn, maar kansen voor waardevolle verbindingen en samenwerkingen.

TurnUp ziet een toekomst voor zich waarin elke geplande interactie wordt gerespecteerd en gewaardeerd, en bijdraagt aan de groei en het succes van zowel individuen als organisaties. Door het probleem van no-shows op wereldwijde schaal aan te pakken, wil TurnUp een blijvende impact hebben op productiviteit, klanttevredenheid en algemeen welzijn.

\subsection{\IfLanguageName{dutch}{Probleem}{Problem}}
TurnUp zal om aan deze missie te werken, in de toekomst gebruik maken van diverse kalenderprogramma's die ze willen integreren in hun eigen kalendersoftware om alle data te centraliseren. Deze gecentraliseerde data zal gemakkelijker toegankelijk zijn voor toekomstig gebruik. Deze integratie kan echter niet zomaar gebeuren aangezien de verschillende softwarepakketten vaak andere manieren hebben van integreren. Dit onderzoek richt zich op het evalueren van de invloed van deze integratie op de gebruikerservaring en het identificeren van innovatieve benaderingen om deze integratie te optimaliseren.
Wanneer het onderzoek is afgewerkt kunnen zij verder aan de slag gaan met de integratie en implementatie van hun kalendersysteem.
Voor andere bedrijven die ook gebruik maken van een of meerdere kalenderapplicaties, kan dit onderzoek ook interessant zijn.

\section{\IfLanguageName{dutch}{Onderzoeksvraag}{Research question}}%
\label{sec:onderzoeksvraag}

Hoe beïnvloedt de integratie van verschillende kalender-API's de gebruikerservaring van ontwikkelaars in softwareapplicaties, en welke innovatieve benaderingen kunnen worden ontwikkeld om de efficiëntie en gebruiksvriendelijkheid van deze integratie te verbeteren?

\section{\IfLanguageName{dutch}{Onderzoeksdoelstelling}{Research objective}}%
\label{sec:onderzoeksdoelstelling}

Het doel van dit onderzoek is om de impact van de integratie van verschillende kalender-API's op de gebruikerservaring te evalueren en om nieuwe methoden te ontwikkelen die de efficiëntie en gebruiksvriendelijkheid van deze integratie verbeteren.

\section{\IfLanguageName{dutch}{Opzet van deze bachelorproef}{Structure of this bachelor thesis}}%
\label{sec:opzet-bachelorproef}

% Het is gebruikelijk aan het einde van de inleiding een overzicht te
% geven van de opbouw van de rest van de tekst. Deze sectie bevat al een aanzet
% die je kan aanvullen/aanpassen in functie van je eigen tekst.

De rest van deze bachelorproef is als volgt opgebouwd:

In Hoofdstuk~\ref{ch:stand-van-zaken} wordt een overzicht gegeven van de stand van zaken binnen het onderzoeksdomein, op basis van een literatuurstudie.

In Hoofdstuk~\ref{ch:methodologie} wordt de methodologie toegelicht en worden de gebruikte onderzoekstechnieken besproken om een antwoord te kunnen formuleren op de onderzoeksvragen.

In Hoofdstuk~\ref{ch:proof-of-concept} wordt de proof of concept besproken. Hoe de applicaties zijn opgebouwd en wat hiervoor werd gebruikt.

In Hoofdstuk~\ref{ch:analyse-testen} worden de resultaten van de proof of concept besproken.

In Hoofdstuk~\ref{ch:conclusie}, tenslotte, wordt de conclusie gegeven en een antwoord geformuleerd op de onderzoeksvraag. Daarbij wordt ook een aanzet gegeven voor toekomstig onderzoek binnen dit domein.