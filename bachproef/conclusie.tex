%%=============================================================================
%% Conclusie
%%=============================================================================

\chapter{Conclusie}%
\label{ch:conclusie}

Dit onderzoek heeft belangrijke inzichten verschaft in de integratie van diverse kalender-API's, met een directe toepassing op de missie van TurnUp om hun kalendersoftware te optimaliseren door de integratie van verschillende kalenderprogramma's. De bevindingen uit dit onderzoek onderstrepen de potentieel significante verbeteringen in gebruikerservaring die bereikt kunnen worden door de methoden die ontwikkeld zijn om de efficiëntie en gebruiksvriendelijkheid van deze integraties te verbeteren.

\section{Impact op TurnUp}
Voor TurnUp biedt deze Proof of Concept concrete voorbeelden van hoe de integratie van kalenderdiensten kan leiden tot een meer gecentraliseerde en toegankelijke datastructuur. Dit is in lijn met hun doelstellingen om een meer geïntegreerd en efficiënt systeem te ontwikkelen dat beter in staat is om met verschillende softwarepakketten samen te werken. De technieken en benaderingen die in dit onderzoek zijn geïdentificeerd, kunnen direct worden toegepast op de plannen van TurnUp om hun eigen kalendersoftware te verbeteren. Dit zal niet alleen hun interne operaties efficiënter maken, maar ook de gebruikerservaring voor hun klanten aanzienlijk verbeteren.

\section{Toekomstig Werk en Aanbevelingen voor TurnUp}
Gezien de complexiteit en de verschillende integratiemogelijkheden die tijdens het onderzoek naar voren kwamen, wordt aanbevolen dat TurnUp voortbouwt op de huidige bevindingen door verder onderzoek te doen naar specifieke integratietechnieken die kunnen worden geoptimaliseerd voor hun specifieke gebruikscases. Verder zou TurnUp kunnen overwegen om een reeks gebruikerstesten uit te voeren om de praktische toepasbaarheid en de reactie van eindgebruikers op de geïntegreerde systemen te evalueren, wat verdere aanpassingen en verbeteringen kan sturen.

Dit onderzoek heeft ook implicaties voor andere bedrijven die kalenderapplicaties gebruiken en soortgelijke integratie-uitdagingen ondervinden. De geleerde lessen en ontwikkelde methoden kunnen dienen als een waardevolle basis voor elk bedrijf dat streeft naar een betere integratie van heterogene systemen in hun operationele omgeving.

